% !TEX encoding = UTF-8 Unicode
% !TEX TS-program = XeLaTeX
\documentclass[UTF8,12pt,letterpaper,oneside]{amsart}
\usepackage[letterpaper]{geometry}
\geometry{top=1.0in, bottom=1.0in, left=1.0in, right=1.0in}
\usepackage{setspace}
%\doublespacing
\usepackage{hyperref}
%\usepackage{times}
\usepackage{graphicx}
\usepackage{rotating}
\usepackage{multirow}
\usepackage{lineno} 
\usepackage{fancyhdr}
\usepackage{hanging}
\pagestyle{fancy}
\rhead{\textsc{Yang} \thepage} 
\renewcommand{\headrulewidth}{0pt} 
\renewcommand{\footrulewidth}{0pt} 
\setlength\headsep{0.333in}
\usepackage{tikz}
%\usepackage{shapes,backgrounds}

\newenvironment{workscited}{\newpage \begin{center} Works Cited \end{center}}{\newpage }

\begin{document}

\noindent Luke \textsc{Yang}\\
\texttt{1004-3383-46}\\
CSCI 170 MW afternoon lecture\\
\textsc{Professor} Aaron \textsc{Cote}\\
Due Thursday, February 20, 2014\\
Homework \#3\\

2. 

(a) $c_1 = a_1 \oplus b_1$

(b) 

\begin{tabular}{ccc}
$a_1$ & $b_1$ & $c_1$ \\\hline
T   & T   & F\\\hline
T   & F   & T\\\hline
F   & T   & T\\\hline
F   & F   & F\\\hline
\end{tabular}

(c) $c_2 = (a_1 \land b_1) \oplus (a_2 \oplus b_2)$

(d) $c_3 = (a_2 \land b_2) \lor (\lnot a_2 \land a_1 \land b_2 \land b_1) \lor (a_2 \land a_1 \land \lnot b_2 \land b_1)$

3.

(a)

(i) $I(\mathrm{You, CSCI\ 170})$

(ii) $\lnot(\mathrm{You,\ CSCI\ 170})$

(iii) You don't think that CSCI 170 is interesting.

(b)

(i) $\exists x \forall y I(x, y)$

(ii) $\lnot \exists x \forall y I(x, y) = \forall x \exists \lnot y I(x, y)$

(iii) For every student there is a class such that the student doesn't think this class is interesting.

(c)

(i) $\forall x \exists y (H(x, y) \land I(x, y))$

(ii) $\lnot \forall x \exists y H(x, y) \land I(x, y) = \exists x \forall y \lnot (H(x, y) \land I(x, y))$

(iii) There is a student who thinks no class in which he enrolls is interesting.

(d)

(i) $\forall x \exists y_1 \exists y_2 (H(x, y_1) \land H(x, y_2) \land I(x, y_1) \land \lnot I(x, y_2))$

(ii) $\lnot \forall x \exists y_1 \exists y_2 (H(x, y_1) \land H(x, y_2) \land I(x, y_1) \land \lnot I(x, y_2)) = \exists x \forall y_1 \forall y_2 \lnot(H(x, y_1) \land H(x, y_2) \land I(x, y_1) \land \lnot I(x, y_2))$

(iii) There is a student for which for each pair of two different classes he is enrolled in he is interested by both or none.

4.

(a) 

(i)

$a$: There are bears in Springfield.

$b$: This rock keeps tigers away.

(ii) 

Premises: $a$. $b$.

Conclusion: $\lnot a \rightarrow b$.

(iii) This is valid.

(iv)

1. $b$ Premise

2. $a \lor b$ Domination by 1

3. $\lnot a \rightarrow b$ Implication of 2 \hfill \textsc{q.e.d.}

(b)

(i)

$a$: I have bacon for breakfast.

$b$: I have salad for lunch.

$c$: I wake up late.

(ii)

Premises: $a \rightarrow b$. $c \rightarrow \lnot a$. $\lnot b$.

Conclusion: $\lnot c$.

(iii) This is not valid.

(iv) We can set $a$ to be false, $b$ to be false, and $c$ to be true, where the premises are all true but the conclusion is false.

(c)

(i)

$a$: I play video games.

$b$: My homework is finished.

$c$: I ace the exam.

$d$: I get an A.

(ii)

Premises: $a \rightarrow b$. $b \rightarrow c$. $\lnot \lnot c \rightarrow d$. $a$.

Conclusion: $d$.

(iii) This is valid.

(iv)

1. $a \rightarrow b$ Premise

2. $b \rightarrow c$ Premise

3. $\lnot \lnot c \rightarrow d$ Premise

4. $c \rightarrow d$ Double negation on 3

5. $a$ Premise

6. $b$ Modus ponens of 5 and 1

7. $c$ Modus ponens of 6 and 2

8. $d$ Modus ponens of 7 and 4 \hfill \textsc{q.e.d.}

(d)

(i)

$a$: Dogs have tails.

$b$: Cats have tails.

$c$: Elephants are pink.

$d$: Elephants are purple.

(ii)

Premises: $a \lor b$. $\lnot a \lor c$. $\lnot b \lor d$.

Conclusion: $c \lor d$.

(iii) This is valid.

(iv)

1. $a \lor b$ Premise

2. $\lnot a \lor c$ Premise

3. $\lnot b \lor d$ Premise

4. $\lnot a \rightarrow b$ Implication of 1

5. $a \rightarrow c$ Implication of 2

6. $b \rightarrow d$ Implication of 3

7. $\lnot c \rightarrow \lnot a$ Contraposition of 5

8. $\lnot c \rightarrow b$ Hypothetical syllogism of 7 and 4

9. $\lnot c \rightarrow d$ Hypothetical syllogism of 8 and 6

10. $c \lor d$ Implication of 9 \hfill \textsc{q.e.d.}




\end{document}

