\section{Lecture 7, Wednesday, February 5}

\begin{ex}

$L(x, y) = x\ \mathrm{loves}\ y$. Universe: people.

Not \textit{Everybody loves somebody} = \textit{Somebody loves Nobody}: $$\lnot \forall x \exists y L(x, y) = \exists x \lnot \exists y L(x, y) = \exists x \forall y \lnot L(x, y)$$

Not \textit{Somebody loves everybody} = \textit{Everybody doesn't love somebody}: $$\lnot \exists x \forall y L(x, y) = \forall x \lnot \forall y L(x, y) = \forall x \exists y \lnot L(x, y)$$

\end{ex}

\begin{ex}

$has(x, y)$, $Universe(x) = $\textit{dogs}, $Universe(y) = $\textit{tails}

$\forall x \exists y {has} (x, y)$: \textit{Every dog has a tail}

$\forall y \exists x {has} (x, y)$: \textit{Every tail is on a dog}

$\exists y \forall x {has} (x, y)$: \textit{Every dog shares a tail}

$\exists x \forall y {has} (x, y)$: \textit{There is a dog which has every tail}

\end{ex}

\begin{ex}

$P(p, h) = p$ \textit{lives in} $h$

$F(s, p) = s$ \textit{is friends with} $p$

$Universe(p) = $\textit{people}, $Universe(h) = $\textit{residence halls}, $Universe(s) = $\textit{students in 170}

\textit{There is a student in this class who is friends with at least one person from every residence hall}: $\exists s \forall h \exists p [F(s, p) \land P(p, h)]$

\end{ex}

\begin{remark}

\textit{There exists a person $s$ in this class such that for every residence hall there exists a person $p$ who is friends with $s$.}

\textit{For every residence hall you can find a person who is friends with someone in this class}$= \forall h \exists p \exists s [F(s, p) \land P(p, h)]$ is different. In this statement $s$ can change, while in the previous (correct) statement $s$ is fixed. 

\textit{There exists a person} $s$ and \textit{there exists a person} $p$ \textit{such that for every residence hall $s$ is friends with $p$ and $p$ is from this residence hall} is a stronger statement. $\exists s \exists p \forall h [F(s, p) \land P(p, h)] \rightarrow \exists s \forall h \exists p [F(s, p) \land P(p, h)]$

Generally, the farther right the $\forall$ quantifier, the stronger the statement.

\end{remark}

\begin{ex}

$Q(x, y, z): x + y = z$, Universe $= \mathbf{Z}$

$\forall x \forall y \exists z Q(x, y, z)$ True

$\exists x \exists y \forall z Q(x, y, z)$ False

$\exists z \forall x \forall y Q(x, y, z)$ False

$\forall z \forall x \forall y Q(x, y, z)$ True

\end{ex}

\begin{remark}

Swapping two adjacent $\forall$ or $\exists$ does not change the statement

\end{remark}

\begin{ex}

$F(x, y) = x$ \textit{is friends with} $y$

$\exists x \forall y \forall z [(F(x, y) \land F(x, z) \land (y \neq z)) \rightarrow \lnot F(y, z)]$: \textit{There is someone for whom all of his friends dislike each other.}

\end{ex}

\begin{defn}

Argument: A sequence of statements starting with assumption (premises) and ending with a conclusion.

\end{defn}

\begin{defn}

Valid: It is impossible for the premise to be true and the conclusion to be false.

\end{defn}

\begin{ex}

If \textit{all roads lead to Rome} and \textit{the 405 does not lead to Rome}, then \textit{the 405 is not a road} is valid.

\end{ex}

\begin{ex}

If \textit{all professors are absent-minded} and \textit{Aaron is absent-minded}, then \textit{Aaron is a professor} is invalid.

\end{ex}

\begin{remark}

Any argument that has one contradiction in its premises is valid.

\end{remark}

Rules of Inference:

Modus ponens: if $p$ and $p \rightarrow q$ then $q$

Modus tollens: if $p \rightarrow q$ and $\lnot q$ then $\lnot p$

Hypothetical syllogism: if $p \rightarrow q$ and $q \rightarrow r$ then $p \rightarrow r$

Simplificaiton: if $p \land q$ then $p$

Resolution: if $p \lor q$ and $\lnot p \lor r$ then $q \lor r$

Use rules of inference to construct more complicated arguments.

More on Table 1 on page 72.

\begin{ex}

If \textit{$\lnot$ Sunny $\land$ cold, swim $\rightarrow$ sunny, $\lnot$ swin $\rightarrow$ canoe, }and\textit{canoe $\rightarrow$ home by sunset.}

Prove \textit{home by sunset.}

\end{ex}

\begin{proof}

1. \textit{$\lnot$ Sunny $\land$ cold}: premise

2. \textit{swim $\rightarrow$ sunny}: premise

3. \textit{$\lnot$ swin $\rightarrow$ canoe}: premise

4. \textit{canoe $\rightarrow$ home by sunset}: premise

5. \textit{$\lnot$ sunny}: simplification 1

6. \textit{$\lnot$ swin}: modus tollens on 2, 5

7. \textit{canoe}: modus ponens on 6, 3

8. \textit{home by sunset}: modus ponens on 7, 4

\end{proof}




