\section{Lecture 8, Monday, February 10}

\begin{defn}

Propositional Equivalences $\equiv$

\end{defn}

\begin{ex}

$(p \rightarrow q) \equiv (\lnot q \rightarrow \lnot p)$ contrapositive

\end{ex}

\begin{ex}

$(p \rightarrow q) \not \equiv (q \rightarrow p)$ converse

\end{ex}

Tautology: $p \lor \lnot p \equiv$ True

Contradiction: $p \land \lnot p \equiv$ False

Implication: $(p \rightarrow q) \equiv (\lnot p \lor q)$

De Morgan's: $\lnot (p \land q) \equiv \lnot p \lor \lnot q$, $\lnot (p \lor q) \equiv \lnot p \land \lnot q$

More on Tables 6, 7, 8 on pages 27 - 28

\begin{prob}

If \textit{you send me an email}$(e)$, then \textit{I will write my code}$(c)$. If \textit{you don't send me an email}, then \textit{I will go to bed}$(b)$. If \textit{I go to bed}, then \textit{I will wake up refreshed}$(r)$.

Prove: if \textit{I don't write my code}, then \textit{I will wake up refreshed}($\lnot c \rightarrow r$).

\end{prob}

\begin{proof}

1: $e \rightarrow c$ premise

2: $\lnot e \rightarrow b$ premise

3: $b \rightarrow r$ premise

4: $\lnot c \rightarrow \lnot e$ contrapositive of 1

5: $\lnot e \rightarrow b$ hypothetical syllogism on 2, 4

6: $\lnot c \rightarrow r$ hypothetical syllogism on 5, 3

\end{proof}


\begin{prob}

Prove: $\lnot [p \lor (\lnot p \land q)] \equiv (\lnot p \land q)$

\end{prob}

\begin{proof}

1: $\lnot [p \lor (\lnot p \land q)]$ premise

2: $\lnot p \land \lnot (\lnot p \land q)$ De Morgan's on 1

3: $\lnot p \land (\lnot \lnot p \lor \lnot q)$ De Morgan's on 2

4: $\lnot p \land (p \lor \lnot q)$ double negation on 3

5: $(\lnot p \land p) \lor (\lnot p \land \lnot q)$ distributive on 4

6. $\text{False} \lor (\lnot p \land \lnot q)$ contradiction on 5

7. $(\lnot p \land \lnot q)$ identity on 6

\end{proof}

Distributive: $p \land (q \lor r) \equiv (p \land q) \lor (p \land r)$, $p \lor (q \land r) \equiv (p \lor q) \land (p \lor r)$

Identity: $p \lor$ False $\equiv p$, $p \land$ True $\equiv p$

Domination: $p \land$ False $\equiv F$, $p \lor$ True $\equiv T$

Implication: $p \rightarrow q \equiv \lnot p \lor q$

Double negation: $\lnot \lnot p \equiv p$

Tautology: $p \lnot \not p \equiv$ True

Associative: $(p \lor q) \lor r \equiv p \lor (q \lor r)$, $(p \land q) \land r \equiv p \land (q \land r)$

\begin{prob}

Prove that $\lnot p \rightarrow (p \rightarrow q) \equiv$ True is tautology.

\end{prob}

\begin{proof}

1. $\lnot p \rightarrow (p \rightarrow q)$ premise

2. $\lnot \lnot p \lor (p \rightarrow q)$ implication on 1

3. $p \lor (p \rightarrow q)$ double negation on 2

4. $p \lor (\lnot p \lor q)$ implication on 3

5. $(p \lor \lnot p) \lor q$ associative on 4

6. True $\lor q$ tautology on 5

7. True domination on 6

\end{proof}

\begin{defn}

Universal instantiation: If $\forall x P(x)$, then $P(c)$ for any $c$

\end{defn}

\begin{defn}

Existential generalization: If $P(c)$, then $\exists x P(x)$

\end{defn}

The two above are easy to use. The following two are dangerous.

\begin{defn}

Universal generalization: If $P(c)$ for any $c$, then $\forall x P(x)$

\end{defn}

\begin{defn}

Existential instantiation: If $\exists x P(x)$, then $P(c)$, specific $c$. \textit{There's no restriction on $c$.}

\begin{ex}

Prove that $\forall x (P(x) \land Q(x)) \equiv (\forall x P(x)) \land (\forall x Q(x))$

\end{ex}

\begin{proof}

1. $P(c) \land Q(c)$ instantiation on the left side

2. $(\forall x P(x)) \land Q(c)$ generalization on 1

3. $(\forall x P(x)) \land (\forall x Q(x))$ generalization on 2

\end{proof}

\begin{prob}

Prove that $\exists x (P(x) \lor Q(x)) \equiv (\exists x P(x)) \lor (\exists x Q(x))$

\end{prob}

\begin{proof}

1. $P(c) \lor Q(c)$ instantiation on the left side

2. $(\exists x P(x)) \lor Q(c)$

3. $(\exists x P(x)) \lor (\exists x Q(x))$

4. $P(c) \lor (\exists x Q(x))$

5. $P(c) \lor Q(d)$

6. If $P(c)$ is true, then $\exists x (P(x) \lor Q(x))$

7. If $Q(d)$ is true, then $\exists x (P(x) \lor Q(x))$

\end{proof}

\begin{prob}

Prove that $\exists x (P(x) \land Q(x)) \not \equiv (\exists x P(x)) \land (\exists x Q(x))$

\end{prob}

\begin{proof}

1. $P(c) \land Q(c)$

2. $(\exists x P(x)) \land Q(c)$

3. $(\exists x P(x)) \land (\exists x Q(x))$

4. $P(c) \land (\exists x Q(x))$

5. $P(c) \land Q(d) \not \equiv \exists x (P(x) \land Q(x))$

\end{proof}

Common fallacies:

Affirming the conclusion: $p \rightarrow q$ and $q$, therefore $p$

Denying the hypothesis: $p \rightarrow q$ and $\lnot p$, therefore $\lnot q$

Circular reasoning/Begging the question: trying to prove the conclusion $q$ while inside the proof $q$ is assumed to be true.

General proof tips:

1. Convince yourself by running through examples;

2. Keep deriving stuff, do not give up;

3. Check your proof very carefully.





\end{defn}

