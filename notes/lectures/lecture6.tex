\section{Lecture 6, Monday, February 3}

\begin{ex}

\textit{You cannot ride the roller coaster if you are under 4 feet tall unless you are older than 16 years.}

$X$ unless $Y$: $(\lnot Y) \rightarrow X$

If you are not older than 16, then (if you are under 4 ftm then you cannot rider), unless you are older than 16. $(\lnot r) \rightarrow \left [ q \rightarrow ( \lnot p ) \right ]$

Truth table:

\begin{tabular}{ccc|c|c|c|c}
$p$ & $q$ & $r$ & $(\lnot r) \rightarrow \left [ q \rightarrow ( \lnot p ) \right ]$ & $(\lnot p) \rightarrow q$ & $q \rightarrow (\lnot p)$ & $(\lnot r) \rightarrow q$\\\hline
 T  &  T  &  T  &  T  &  T  &  F  &  T\\
 T  &  T  &  F  &  F  &  T  &  F  &  T\\
 T  &  F  &  T  &  T  &  T  &  T  &  T\\
 T  &  F  &  F  &  T  &  T  &  T  &  F\\
 F  &  T  &  T  &  T  &  T  &  T  &  T\\
 F  &  T  &  F  &  T  &  T  &  T  &  T\\
 F  &  F  &  T  &  T  &  F  &  T  &  T\\
 F  &  F  &  F  &  T  &  F  &  T  &  F
\end{tabular}

Some equivalent expressions:

$(p \land q ) \rightarrow r$

$(q \land \lnot r) \rightarrow \lnot p$

$q \rightarrow (\lnot p \lor r)$

\end{ex}

Propositional logic are digital logic are one-on-one correspondent. Every propositional logic expression has its unique digital logic expression, vice versa. 

First order logic: propositional logic + predicates.

\begin{defn}

A predicate is a sentence with one or more variables with becomes a proposition when specific values substituted for variables.

\end{defn}

We write predicate as a function.

\begin{ex}

$P(n) = n$ \textit{is an odd integer}.

$P(5) = 5$ \textit{is an odd integer}.

$P(6) = 6$ \textit{is an odd integer}.

\end{ex}

You may need to specify \textit{universe}.

\begin{ex}

$P(x) = x$ \textit{is larger than cake}. Universe: food items.

$P(\text{pie})$: \textit{pie is larger than cake.}

\end{ex}

\begin{ex}

$P(x) = x$ \textit{is mortal}. Universe: humans.

\textit{All humans are mortal}: $\forall x P(x)$

\textit{There is a human who is mortal}: $\exists x P(x)$

\textit{No humans are mortal}: $\forall \lnot P(x)$

\textit{There is a human who is not mortal}: $\exists x \lnot P(x)$

\end{ex}

\begin{ex}

$E(x) = x$ \textit{is an elephant}

$P(x) = x$ \textit{is pink}. Universe: everything.

$\forall x [ E(x) \land  P(x) ]$: \textit{everything is pink elephant.}

$\forall x [ E(x) \rightarrow P(x) ]$: \textit{all elephants are pink.}

$\exists x [ E(x) \land P(x) ]$: \textit{There is a pink elephant.}

$\exists x [ E(x) \rightarrow P(x) ]$: \textit{There is something which is either pink or not an elephant.}

\end{ex}

\begin{remark}

$\exists x [ E(x) \rightarrow P(x) ] = \exists x [ \lnot E(x) \lor P(x) ]$

\end{remark}

\begin{ex}

\textit{All humans are mortal} and \textit{There is a human who is not mortal} are logically opposite.

\textit{There is a human who is mortal} and \textit{No humans are mortal} are logically opposite.

\end{ex}

$\lnot \forall x P(x) = \exists x \lnot P(x)$

$\lnot \exists x P(x) = \forall x \lnot P(x)$

\begin{remark}

This is actually De Morgan's Law, since $\forall$ is the combination of a lot of $\land$, and $\exists$ is equivalent to $\lor$.

\end{remark}

\begin{ex}

$\lnot \forall x [ E(x) \rightarrow P(x) ] = \exists x \lnot [ E(x) \rightarrow P(x) ]$

\end{ex}

\begin{ex}

\textit{All that glitters is not gold} literally translates to $\forall x [ {Gl}(x) \rightarrow \lnot {Go}(x) ]$, but it actually means $\exists x [ \lnot {Go}(x) \land {Gl}(x) ]$. Ambiguity of English.

\end{ex}

\begin{ex}

$L(x, y) = x\ \mathrm{loves}\ y$. Universe: people.

\textit{Everybody loves somebody}: $\forall x \exists y L(x, y)$

\textit{Somebody loves everybody}: $\exists x \forall y L(x, y)$

\textit{There is somebody whom everybody loves}: $\exists y \forall x L(x, y)$

\textit{Everybody is loved by somebody}: $\forall y \exists x L(x, y)$

\end{ex}

