\section{Lecture 2, Thursday, January 15}

\begin{ex}
3 Truth machines are in stock. A machine corresponds true/false to red/green, but patterns for different machines can be different. 1 machine is broken (it answers arbitrarily) and 2 are working. Ask one single question (with a single yes/no answer) to one machine and determine which two are working.
\end{ex}

\begin{soln}
Ask M1: is it the case that exactly one of these is true: Red means yes; 2nd
machine is broken.

If Red, choose M2; if Green, choose M3.

Assume M1 works, and red means yes. Then it answers red if the 2nd machine
works.

Assume M1 works, and green means yes. Then it answers green if the 2nd machine
is broken.

Assume M1 works, both M2 and M3 work. Choosing either would be good.


Another method: meta-question: If I'd asked you, ``Is machine 2 broken?'', would
you answer green?
\end{soln}

\begin{defn}

The cardinality of a set $A$, denoted by $|A|$, is the number of distinct elements in $A$.

\end{defn}

\begin{ex}
$|\{\text{cake, pie, cake}\}| = 2$
\end{ex}
\begin{ex}
$|\O| = |\{\}| = 0$
\end{ex}
\begin{ex}
$|\{\O\}| = 1$
\end{ex}
\begin{ex}
$|\{\mathbf{Z}, \mathbf{N}\}| = 2$
\end{ex}
\begin{ex}
$|\mathbf{Z}| = \infty$
\end{ex}
\begin{defn}

The power set of $S$, denoted by $P(S)$, is the set of all subsets of $S$.

\end{defn}

\begin{ex}
$S = \{\text{cake, pie}\}$, $P(S) = \{ \O, \{\text{cake}\}, \{\text{pie}\}, \{\text{cake, pie}\} \}$
\end{ex}

\begin{ex}
$P(\O) = \{ \O \}$
\end{ex}

\begin{ex}
$P(\{ \O \}) = \{ \O, \{ \O \} \}$
\end{ex}

\begin{ex}
$P(P(\{a\})) = P(\{ \O, \{ a \} \}) = \{ \O, \{ \O \}, \{ \{ a \} \}, \{ \O, \{ a \} \} \}$
\end{ex}

$P(A) = P(B) \rightarrow A = B$

$|A| = n \rightarrow |P(A)| = 2^n$


\begin{defn}
An ordered $n$-tuple is a collection of elements where order matters, i.e., an ordered set.
\end{defn}

\begin{ex}
$(a, b) \neq (b, a)$
\end{ex}

\begin{ex}
$(a, a, b) \neq (a, b)$
\end{ex}

\begin{defn}
The Cartesian product of two sets $A$ and $B$, denoted by $A \times B$, is an unordered set that consists all ordered pairs of $(a, b)$ such that $a$ is in $A$ and $b$ is in $B$.

$$A \times B = \{ (a, b) | a \in A \land b \in B\}$$
\end{defn}

\begin{ex}
$\{1, 2\} \times \{1, 3, 4\} = \{(1, 1), (1, 3), (1, 4), (2, 1), (2, 3), (2, 4) \}$
\end{ex}

\begin{ex}
$\{(1, 1), (1, 3), (1, 4), (2, 1), (2, 3), (2, 4) \} \times \{1, 2\} = \{((1, 1), 1), ((1, 3), 1), \cdots\}$
\end{ex}

\begin{defn}
A function for $A$ to $B$, denoted by $f: A \rightarrow B$, takes as input an element from set $A$ and outputs an element from set $B$.
\end{defn}

\begin{defn}
A function $f: A \rightarrow B$ is injective or one-to-one if every input maps to a distinct output, i.e., for an injective function $f$, $f(a) = f(b) \rightarrow a = b$
\end{defn}

\begin{ex}
$f(x): \mathbf{R} \rightarrow \mathbf{Z}, f(x) = \lfloor x \rfloor$ is not injective.
\end{ex}

\begin{remark}
Floor function $\lfloor x \rfloor$, ceiling function $\lceil x \rceil$.
\end{remark}

\begin{defn}
A function $f: A \rightarrow B$ is surjective or onto if every element in $B$ can be produced.
\end{defn}

\begin{ex}
$f(x): \mathbf{R} \rightarrow \mathbf{Z}, f(x) = \lfloor x \rfloor$ is surjective.
\end{ex}

\begin{defn}
A function $f$ is bijective or one-to-one correspondence if it's both injective and surjective.
\end{defn}

\begin{ex}
$f(x): \mathbf{Z} \rightarrow \text{even integers}, f(x) = 2x$ is both injective and surjective, thus is bijective.
\end{ex}

\begin{defn}
A sequence is a function from $\mathbf{N}$ to some set $S$.
\end{defn}

\begin{ex}
$f_0 = 0, f_1 = 1, f_2 = 2, f_3 = 3, f_4 = 5, f_5 = 8, \cdots$
\end{ex}

\begin{ex}
$1, 2, 3, 4, \cdots$
\end{ex}

\begin{ex}
$1, 4, 9, 16, \cdots$
\end{ex}

\begin{ex}
$f_n = f_{n - 1} + f_{n - 2}$, $f_0 = 0$, $f_1 = 1$
\end{ex}

\begin{remark}
Recurrence relations: recursive definitions of sequences.
\end{remark}

\begin{ex}
3, 3, 3, 3, $\cdots$: $f_0 = 3$, $f_n = f_{n - 1}$
\end{ex}

\begin{ex}
$f_n = 2n$: $f_n = 2 + f_{n - 1}$
\end{ex}

\begin{ex}
$f_n = n^2$: $f_n = f_{n - 1} + 2n - 1$
\end{ex}

\begin{ex}
$f_n = n + (-1)^n$: $f_n = f_{n - 2} + 2$
\end{ex}

\begin{ex}
The polulation of world in 2010 is 6.9 billion, assume it grows at an annual rate of 11\%.
$f_n = 1.011 f_{n - 1}$, $f_0 = 6.9$billion
\end{ex}

