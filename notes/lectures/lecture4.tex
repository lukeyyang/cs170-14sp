\section{Lecture 4, Monday, January 27}

Linear search $\Theta(n)$

Binary search $\Theta(\log n)$

\begin{remark}
$\log n$ means $\log_2 n$ in computer science, unless otherwise specified.
\end{remark}

Selection sort $\Theta(n^2)$

Insertion sort $\Theta(n^2)$

Quick sort $\Theta(n^2)$ (worst case) $\Theta(n\log n)$ (Average case)

Merge sort $\Theta(n \log n)$

Radix sort $\Theta(kn)$ where $k$ stands for the number of digits in the sorted numbers

\begin{ex}
Two functions $f(n) = \log_{10} n$ and $g(n) = \log_2 n = 2 \log n$. Apparently $f(n) = \mathrm{O}(g(n))$. True or false: $f(n) = \Theta(g(n))$.

True, since $\log_{10} n = \dfrac{\log_2 n}{\log_2 10}$.
\end{ex}

\begin{ex}
$\log^2 n = (\log_n)^2$. Apparently $\log_n = \mathrm{O}(\log^2 n)$. True or false: $\log_n = \Omega(\log^2 n)$.

False, since $\log^2 n = \log n \cdot \log n$.
\end{ex}

\begin{ex}
It is true that $\log n = \Theta(\log n^2)$.
\end{ex}

\begin{ex}
Although for any given number $n$, the value of $10 \log^{100} n$ is larger than the value of $\dfrac{1}{100} \sqrt{n}$, the rate of growth of $10 \log^{100} n$ is smaller than that of $\dfrac{1}{100} \sqrt{n}$. The former is polylogarithmic ($\log^c n$) and the latter is polynomial ($n^d$). $\log^c n = \mathrm{O} (n^d)$.
\end{ex}

\begin{ex}
Although for any given number $n$, the value of $100 n^100$ is larger than the value of $1.01^{\tfrac{n}{100}}$, the rate of growth of $100 n^100$ is smaller than that of $1.01^{\tfrac{n}{100}}$. The former is polynomial ($n^d$) and the latter is exponential ($c^{\tfrac{n}{e}}$). $n^d = \mathrm{O} (c^{\tfrac{n}{e}})$, for $c > 1$.
\end{ex}

Hierarchy of classifications of the rate of growth of a function:

Constant $\Theta(1)$

Polylogarithmic $\mathrm{O} (\log^c n)$

Polynomial $\mathrm{O} (n^c)$

Exponential is not polynomial

$c^{\log n}$ undefined

\begin{ex}
$f_1 = n^n$ is exponential

$f_2 = \log^2 n$ is polylog

$f_3 = n^{1.0001}$ is polynomial

$f_4 = 1.0001^n$ is exponential

$f_5 = 2^{\sqrt{\log n}}$ is undefined

$f_6 = n \log^{1.0001} n$ is polynomial

Since $f_3 = n \cdot n^{0.0001}$ and $f_6 = n \cdot \log^{1.0001} n$, $ n^{0.0001}$ is polynomial and $\log^{1.0001} n$ is polylog, $f_3 > f_6$ in terms of rate of growth.

Since $2^{\sqrt{\log n}} < 2^{\log_2 n} = n$, $f_5 < f_6$ in terms of rate of growth.

Comparing $f_2 = \log^2 n$ and $f_5 = 2^{\sqrt{\log n}}$: $\log f_2 = 2 \log(\log n)$ is polylogarithmic of $\log n$ and $\log f_5 = \sqrt{\log n}$ is polynomial of $\log n$, $\log^2 n = \mathrm{O} (2^{\sqrt{\log n}})$, $f_2 < f_5$ in terms of rate of growth.

Comparison of the rates of growth: $f_2 < f_5 < f_6 < f_3 < f_4 < f_1$
\end{ex}

\begin{ex}
$f_1 = 2^{100n}$ is exponential

$f_2 = 2^{n^2}$ is exponential

$f_3 = 2^{n!}$ is exponential

$f_4 = 2^{2^n}$ is exponential

$f_5 = n^{\log n}$ is undefined

$f_6 = n \log n \log(\log n)$ is polynomial

$f_7 = n^{\tfrac{3}{2}}$ is polynomial

$f_8 = n \log^{\tfrac{3}{2}} n$ is polynomial

$f_9 = n^{\tfrac{4}{3}} \log^2 n$ is polynomial

For the polynomials, take out common factor $n$:

$f_6 = \log n \log(\log n)$ is polylog

$f_7 = n^{\tfrac{1}{2}}$ is polynomial

$f_8 = \log^{\tfrac{3}{2}} n$ is polylog

$f_9 = n^{\tfrac{1}{3}} \log^2 n$ is polynomial

For $f_6$ and $f_8$, take out common factor $\log n$:

$f_6 = \log(\log n)$ is polylog of $\log n$

$f_8 = \log^{\tfrac{1}{2}} n$ is polynomial of $\log n$

For $f_7$ and $f_9$, take out common factor $n^{\tfrac{1}{3}}$

$f_7 = n^{\tfrac{1}{6}}$ is polynomial

$f_9 = \log^2 n$ is polylog

For the polynomials, $f_6 < f_8 < f_9 < f_7$.

For the exponentials, take logarithms:

$f_1 = 100n$ is polynomial (linear)

$f_2 = n^2$ is polynomial (quadratic)

$f_3 = n! = \mathrm{O}(n^n)$ is exponential

$f_4 = 2^n$ is exponential

For the exponentials, $f_1 < f_2 < f_4 < f_3$

For $f_5$ and $f_1$, take logarithms:

$f_5 = \log(n^{\log n}) = \log^2 n$ is polylog

$f_1 = 100n$ is polynomial

$f_6 < f_8 < f_9 < f_7 < f_5 < f_1 < f_2 < f_4 < f_3$
\end{ex}

Running time (seconds, unless otherwise specified):

\begin{tabular}{ccccc}
 & $n = 10$ & $10^2$ & $10^3$ & $10^6$ \\\hline
$\log n$ & $3 \times 10^{-11}$ & $7 \times 10^{-11}$ & $10^{-10}$ & $2 \times 10^{-10}$ \\\hline
$n^2$ & $10^{-9}$ & $10^{-7}$ & $10^{-5}$ & circa $10^{2}$ \\\hline
$2^n$ & $10^8$ & $4 \times 10^{11}$ years &  & 
\end{tabular}

$f(n) = \mathrm{O}(h(n))$, $g(n) = \mathrm{O}(h(n))$. It is intuitively true that $f(n) + g(n) = \mathrm{O}(h(n))$

\begin{proof}
$f(n) = \mathrm{O}(h(n)) \Leftrightarrow f(n) \leq c_0 h(n)$, $\forall n > n_0$,  $\exists c_0, n_0$, $g(n) = \mathrm{O}(h(n)) \Leftrightarrow g(n) \leq c_1 h(n)$, $\forall n > n_1$, $\exists c_1, n_1$, then $f(n) + g(n) \leq (c_0 + c_1) h(n)$, $\forall n > \max (n_0, n_1)$. Set $c = c_0 + c_1$ and $n_2 = \max (n_0, n_1)$, then $f(n) + g(n) \leq c h(n)$, $\forall n > n_2$.
\end{proof}
