\section{Lecture 5, Wednesday, January 29}

\begin{prob}

$f(n) = {\rm O}(h(n))$, $g(n) = {\rm O}(h(n))$, then $f(n) + g(n) = {\rm O} (h(n))$. True or false: $\sum_{i = 1}^n f(n) = {\rm O}(h(n))$.

False. Set $f(n) = n$, $h(n) = n$, then $n = {\rm O}(n)$, but $\sum_{i = 1}^n n = n^2 \neq {\rm O}(n)$.

\end{prob}

\begin{remark}

The summation of $f(n)$ for any constant number $c$ times $\sum_{i = 1}^c = {\rm O}(f(n))$, but in this case $n$ is a variable, which makes the statement false.

\end{remark}

\begin{prob}

$f(n) = {\rm O}(s(n))$, $g(n) = {\rm O}(r(n))$, True or false: $f(n) - g(n) = {\rm O}(s(n) - r(n))$.

False. Set $f(n) = n^2$, $g(n) = 1$, $s(n) = n^2$, $r(n) = n^2 - 1$, then $n^2 = {\rm O}(n^2)$, and $1 = {\rm O}(n^2 - 1)$, but $n^2 - 1 \neq {\rm O}(1)$

\end{prob}

\begin{remark}

$f(n) \leq c_0 s(n)$ for $n > n_0$ and $g(n) \leq c_1 r(n)$ for $n > n_1$ can't lead to $f(n) - g(n) \leq c [s(n) - r(n)]$.

$f(n) - g(n) \leq c_0 s(n) - g(n) \geq c_0 s(n) - c_1 r(n)$. No way to yield a relation between the left side and the right side.

\end{remark}

\begin{prob}

$f(n) = \Omega(s(n))$, $g(n) = \Omega(r(n))$, True or false: $f(n) - g(n) = \Omega(s(n) - r(n))$.

False. Set $f(n) = n + 1$, $g(n) = n$, $s(n) = n$, $r(n) = 1$, then $n + 1 = \Omega(n)$, and $n = \Omega(1)$, but $1 \neq \Omega(n - 1)$

\end{prob}

\begin{prob}

$f(n) = \Theta(s(n))$, $g(n) = \Theta(r(n))$, True or false: $f(n) - g(n) = \Omega(s(n) - r(n))$.

False. Set $f(n) = n^2 + n$, $g(n) = n^2$, $s(n) = n^2 + 1$, $r(n) = n^2$, then $n^2 + n = \Theta(n^2 + 1)$, and $n^2 = \Theta(n^2)$, but $n \neq \Theta(1)$

\end{prob}

Propositional Logic: The building block of proper logic is the proposition.

\begin{defn}

Proposition: a statement of fact which is either true or false.

\end{defn}

\begin{ex}

\textit{36 is a prime number} is a proposition.

\textit{Spaghetti grows on trees} is a proposition.

\textit{Go play Hearthstone} is not a proposition.

\textit{I am bored by this class} is a proposition.

\textit{The cake is a lie} is a proposition.

$1 + 1 = 2$ is a proposition.

$x + 1 = 2$ is not a proposition. It is a \textit{predicate}.

\textit{This statement is false} is not a proposition. It is neither true nor false.

\end{ex}

\begin{ex}

$p =$ \textit{there is life on Mars.}

$q =$ \textit{there is life on Earth.}

$\lnot p =$ \textit{no life on Mars.}

$p \land q =$ \textit{life on both}

$p \lor q$

\begin{tabular}{cccccc}
$p$ & $q$ & $\lnot p$ & $p \land q$ & $p \lor q$ & $p \oplus q$\\\hline
  T &   T &         F &           T &          T &           F \\
  T &   F &         F &           F &          T &           T \\
  F &   T &         T &           F &          T &           T \\
  F &   F &         T &           F &          F &           F
\end{tabular}

\end{ex}

$\lnot$ is an unary operator. It only takes one argument.

Theoretically there are 16 binary operators, including $\land$, $\lor$, $\oplus$.

\begin{defn}

$p \rightarrow q$: $\lnot p \lor q$, if $p$ then $q$, $p$ implies $q$, $p$ only if $q$, $q$ if $p$, $q$ is necessary for $p$, $p$ is sufficient for $q$   

\end{defn}

\begin{ex}

\texttt{If Obama is president then }$2 + 2 = 4$ is true, true.

\texttt{If Obama is president then }$2 + 2 = 4$ is true, false.

\texttt{If Spaghetti grows on trees then }$2 + 2 = 4$ is false, true.

\texttt{If Spaghetti grows on trees then }$2 + 2 = 5$ is false, false.

\end{ex}

\begin{defn}

Implication: $p \rightarrow q$

\end{defn}

\begin{defn}

Converse: $q \rightarrow p$

\end{defn}

\begin{defn}

Contraposition: $\lnot q \rightarrow \lnot p$

\end{defn}

\begin{theorem}

$p \rightarrow q$ and $\lnot q \rightarrow \lnot p$ are equivalent.

\end{theorem}

\begin{defn}

$p \Leftrightarrow q$: $p$ implies $q$ and $q$ implies $p$, $p$ if and only if $q$, $p$ iff $q$, $\lnot(p \oplus q)$

\end{defn}

\begin{tabular}{ccccccc}
$p$ & $q$ & $\lnot p$ & $\lnot q$ & $\lnot q \rightarrow \lnot p$ & $p \rightarrow q$ & $p \Leftrightarrow q$\\\hline
  T &   T &         F &           F &          T &           T & T \\
  T &   F &         F &           T &          F &           F & F \\
  F &   T &         T &           F &          T &           T & F\\
  F &   F &         T &           T &          T &           T & T
\end{tabular}

\begin{remark}

In common English, people use \textit{if} a lot but actually they mean \textit{iff}.

\end{remark}

Order of operation: first negation, then everything else.

\begin{defn}

A set of compound propositions is consistent if it is possible to assign True or False values to all atomic propositions such that all compound propositions are true.

\end{defn}

\begin{ex}

\textit{You will get an A in this class} or \textit{you will die trying},

If \textit{you die trying}, \textit{you will (posthumously) get an A in this class},

\textit{You will get an A in this class} and not \textit{die trying}

are consistent.

\end{ex}

\begin{ex}

\textit{You will get an A in this class} or \textit{you will die trying}.

If \textit{you die trying}, \textit{you will (posthumously) get an A in this class}.

\textit{You will get an A in this class}.

are inconsistent.

\end{ex}

\begin{prob}

Two doors, each has a sign. One prize behind one of the two doors. One sign is true, the other one is false. Sign 1: \textit{There is a prize behind this door and no prize behind other}. Sign 2: \textit{There is a prize behind one door and nothing behind the other}.

The prize is behind door 2. When sign 1 is true, sign 2 must be true, but there's only one true sign, then sign 1 is false and sign 2 is true. Since sign 1 is false, the prize is behind door 2.

$r = $ \textit{There is a prize behind door 1}.

$s = $ \textit{There is a prize behind door 2}.

$p = $ \textit{Sign 1 is true}.

$q = $ \textit{Sign 2 is true}.

Premise: $p \oplus q$

Sign 1: $p \Leftrightarrow (r \land \lnot s)$

Sign 2: $q \Leftrightarrow (r \oplus s)$



\end{prob}
















