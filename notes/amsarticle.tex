%%
% Sample article using the amsart style
% Based on an example provided by AMS
% modified somewhat for use at NIU Math Dept.
%
\documentstyle{amsart}

%%
% BibTeX logo; normally not needed
\def\BibTeX{{\rm B\kern-.05em{\sc i\kern-.025em b}\kern-.08em
    T\kern-.1667em\lower.7ex\hbox{E}\kern-.125emX}}

%%
\hfuzz1pc % Don't bother to report overfull boxes if overage is < 1pc

%       Theorem environments
% theorem numbering gets reset to 1 in each section
\newtheorem{thm}{Theorem}[section]
% most others are numbered together with theorems
\newtheorem{cor}[thm]{Corollary}
\newtheorem{lem}[thm]{Lemma}
\newtheorem{prop}[thm]{Proposition}
\newtheorem{exmp}[thm]{Example}
%%
% just an example of what will happen if you skip the [thm]
% part -- conjectures will be numbered consecutively
\newtheorem{conj}{Conjecture}
\theoremstyle{definition}
\newtheorem{defn}{Definition}[section]
\theoremstyle{remark}
\newtheorem{rem}{Remark}[section]
\newtheorem{notation}{Notation}
\renewcommand{\thenotation}{}  % to make the notation environment unnumbered

%%
% equation counter will be reset at the start of each section
\numberwithin{equation}{section}

%       Math definitions
% Blackboard Bold letters
\newcommand{\C}{\mathbbb{C}}
\newcommand{\N}{\mathbbb{N}}
\newcommand{\Q}{\mathbbb{Q}}
\newcommand{\R}{\mathbbb{R}}
\newcommand{\Z}{\mathbbb{Z}}
% operators
\newcommand{\cov}{\operatorname{cov}}
\newcommand{\cf}{\operatorname{cf}}
\newcommand{\add}{\operatorname{add}}
\newcommand{\non}{\operatorname{non}}
\newcommand{\per}{\operatorname{per}}
\newcommand{\End}{\operatorname{End}}

\begin{document}

\title[AMSTEX/AMSART Sample Paper] % running head version
{Sample Paper for the `AMSTEX' Option\\
and the `AMSART' Documentstyle}

%%
% See the complete version of this file, testart.tex, for more
% complicated examples
%
\author{Umberto Orsini}
\address[Umberto Orsini]{Department of Mathematical Sciences\\
        Northern Illinois University\\
        DeKalb, Illinois 60115}
%% Note the doubled @@:
\email[U.~Orsini]{orsini@@math.niu.edu}
\thanks{Research supported in part by NSF grant
CCR-87-10433 and DARPA Contract N00019-89-J-1988.}

\date{July 25, 1994}

\subjclass{Primary 05C38, 15A15; Secondary 05A15, 15A18}

\maketitle

\begin{abstract}
This paper is a sample illustrating  the use
of the American Mathematical Society preprint
documentstyle, \verb+amsart+, and the use of bibliography
databases. It should be processed with \AmS-\LaTeX\ first,
then ran through \BibTeX, and through \AmS-\LaTeX\ at least
once more.
\end{abstract}

\section{Introduction}
\label{intro}
This paper illustrates the use of the {\tt amsart} documentstyle, as
well as the use of features from the {\tt amstex} option.  It is a 
greatly abbreviated skeleton version of the file {\tt testart.tex}
provided by the \AmS. 

\section{Enumeration of Hamiltonian paths in a graph}
Let $\bold A=(a_{ij})$ be the adjacency matrix of
graph $G$ (\cite{lewi:plur86}). 

The corresponding Kirchhoff matrix $\bold K=(k_{ij})$ is obtained from 
$\bold A$ by replacing in $-\bold A$ each diagonal entry by the degree of its
corresponding vertex (see \cite{chom:barr86}, \cite{barw:mode89}, and
\cite{chel:moda80}); i.e., the $i$th diagonal entry is identified
with the degree of the $i$th vertex.

[Rest of your article goes here]

\bibliographystyle{amsplain}
%%
% requires a BiBTeX file sample.bib
\bibliography{sample}

\end{document}

